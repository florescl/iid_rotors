\documentclass{article}
\usepackage{tikz}
\usetikzlibrary{spy} 
%\usetikzlibrary{decoration} 
\usepackage{pgf,tikz}
\usepackage{pgfplots,comment}
\usetikzlibrary{spy}
\usetikzlibrary{backgrounds}
\usetikzlibrary{decorations}
\date{}
\begin{document}

\pagestyle{empty}

\pgfdeclarelayer{background layer}
\pgfsetlayers{background layer,main}
\definecolor{darkgray}{rgb}{0.25,0.25,0.25}
\definecolor{lightgray}{rgb}{0.75,0.75,0.75}
%
\begin{figure}[h!]

\begin{tikzpicture}
    [scale=0.3,only marks,reddot/.style={fill=red,circle,inner sep=1pt, minimum width=0.5pt},bluedot/.style={fill=blue,circle, inner sep=2pt, minimum width=0.5pt},
    blackdot/.style={fill=black,circle, inner sep=2pt, minimum width=0.5pt},
    yellowdot/.style={fill=yellow,circle, inner sep=2pt, minimum width=0.5pt},
    greendot/.style={fill=green,circle, inner sep=2pt, minimum width=0.5pt},    
     %using the 'spy' to magnify a part of the picture
     spy using outlines={rectangle,lens={scale=3}, size=6cm, connect spies},
     %using the decoration 'brace' (=a curly brace as path replacement)
     %decoration={brace,amplitude=2pt}
]

\input{uni1.tex}


\end{tikzpicture}
%\caption{Red: points from 1st quarter, Blue: points from 3rd quarter, and Matching between them for the same SNW with 40 particles}
\end{figure}
\end{document}
